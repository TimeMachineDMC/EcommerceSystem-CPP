\documentclass[11pt]{article}
\usepackage[UTF8]{ctex}      % 中文支持
\usepackage{amsmath}         % 数学公式
\usepackage{geometry}        % 页面边距
\usepackage{enumitem}        % 自定义列表
\usepackage{graphicx}        % 图片插入
\usepackage{float}           % 图片位置控制
\usepackage{tikz}
\usetikzlibrary{positioning}
\usetikzlibrary{calc}

\usepackage{listings}
\usepackage{xcolor}
\usepackage{ctex} % 保证主文档支持中文

\lstset{
  basicstyle=\ttfamily\small,
  numbers=left,
  numberstyle=\tiny,
  stepnumber=1,
  numbersep=5pt,
  frame=single,
  breaklines=true,
  keywordstyle=\color{blue},
  commentstyle=\color{gray},
  stringstyle=\color{purple},
  tabsize=2,
  showstringspaces=false,
  extendedchars=true,      % 关键:允许中文等扩展字符
  inputencoding=utf8       % 关键:输入文件使用 UTF-8 编码
}

\geometry{a4paper, margin=2.5cm}
\title{电商交易平台设计与实现}
\author{黄梓桐\\2023211045}
\date{2025年6月20日}

\begin{document}

\maketitle
\tableofcontents
\newpage

\section{前言}

随着移动互联网的快速发展,电商平台已经成为人们日常生活不可或缺的一部分。为了加深对面向对象程序设计与实践的理解,本项目基于C++语言,采用面向对象设计思想,分阶段实现了一个功能完善的电商交易平台。

本项目的开发分为三个阶段,每一阶段的实现都在前一阶段的基础上逐步扩展,最终完成了一个具备完整用户管理、商品管理、交易管理和网络通信能力的综合性电商平台。虽然开发流程分为三阶段,但系统设计始终围绕同一个目标题目展开,所有阶段的代码和架构都是连续演进、互相兼容的。

平台主要实现了如下核心需求:

\begin{itemize}
    \item \textbf{用户系统:} 支持新用户注册、用户登录、密码修改、余额查询和充值。用户分为商家和消费者两类,信息持久化保存,权限分明。
    \item \textbf{商品系统:} 支持多种商品类型(如图书、食品、服装),商品信息由商家发布并管理,支持商品增删改查、品类折扣、信息持久化等。
    \item \textbf{购物与订单系统:} 消费者可通过购物车管理拟购买商品,实现订单生成与管理、余额支付、支付后的资金转移、订单状态跟踪、库存冻结与恢复等关键业务逻辑。
    \item \textbf{网络通信与并发处理:} 第三阶段实现了典型的CS结构,客户端与服务器端采用socket进行通信,服务器端支持多线程并发、会话互斥管理(同一账户不可多端登录)、购物车/订单等状态的持久化存储。
\end{itemize}

本系统不仅在功能上覆盖了常见电商平台的基础业务,还在设计和实现过程中严格遵循面向对象的封装、继承和多态思想,采用模块化结构、分层管理、友元函数与接口协议规范等现代编程理念,具备良好的可扩展性与维护性。

通过本实验项目,不仅锻炼了面向对象建模、复杂系统分层设计和C++高级特性的实践能力,也积累了多线程、文件存储、网络通信、协议设计等实际工程经验,为后续更大型项目开发打下坚实基础。

\newpage
\section{总体设计}

本电商交易平台系统采用面向对象的分层和模块化架构设计,整个系统主要划分为五大核心子系统:用户子系统、商品子系统、订单与购物车子系统、网络通信子系统和数据存储子系统。各子系统各司其职,并通过明确定义的接口进行交互,协同完成平台的各项业务功能。

\subsection{子系统划分与功能简介}

\begin{itemize}
    \item \textbf{用户子系统}:负责平台用户的注册、登录、密码与余额管理、用户身份权限控制(顾客/商家)、会话互斥管理等功能。核心类包括 \texttt{User}(基类)、\texttt{Customer}、\texttt{Seller}、\texttt{UserManager}。
    
    \item \textbf{商品子系统}:实现商品的增删查改、商品分类(如图书、食品、服装)、折扣策略、库存管理、商品持久化等功能。核心类包括 \texttt{Product}(基类)、\texttt{Book}、\texttt{Food}、\texttt{Clothes}、\texttt{ProductManager}。
    
    \item \textbf{订单与购物车子系统}:负责顾客购物车的管理、订单的生成与支付、订单取消、库存冻结与恢复、订单和购物车的数据持久化。核心类包括 \texttt{CartItem}、\texttt{ShoppingCart}、\texttt{Order}、\texttt{OrderManager}。
    
    \item \textbf{网络通信子系统}:采用C/S结构,通过Socket和自定义JSON协议实现客户端和服务器之间的消息通信,支持多线程并发、会话唯一性控制。核心类包括 \texttt{SocketStream}、网络协议处理模块等。
    
    \item \textbf{数据存储子系统}:将所有核心业务数据(用户、商品、订单、购物车、会话锁等)以文件形式持久化保存,确保系统重启后数据不丢失。典型数据文件包括 \texttt{users.txt}、\texttt{products.txt}、\texttt{orders.txt}、\texttt{carts/}、\texttt{sessions/} 等。
\end{itemize}

\subsection{子系统之间的关系与协作}

各子系统之间通过明确的接口进行解耦协作,彼此依赖关系如下:

\begin{itemize}
    \item 用户通过客户端登录,所有业务操作请求经由网络通信子系统转发到服务器端;
    \item 用户子系统提供身份认证与权限分配,驱动所有业务行为;
    \item 商品子系统为订单和购物车子系统提供商品信息、库存、价格、折扣等支撑;
    \item 订单与购物车子系统实现用户的购物与交易过程,涉及对商品和用户余额的联动修改;
    \item 数据存储子系统为所有核心数据对象提供持久化支撑,所有变动及时保存至本地数据文件;
    \item 网络通信子系统连接客户端与服务器,保障多用户并发场景下的业务一致性与互斥性。
\end{itemize}

\subsection{系统架构关系图}

系统结构关系如下图所示:

\begin{figure}[htbp]
\centering
\begin{tikzpicture}[>=latex, font=\small]

% 顶层 客户端 / 服务器端
\node (client) at (-5, 0) [draw, rounded corners, fill=gray!10, minimum width=2.8cm, minimum height=1cm, align=center] {Client端};
\node (server) at (5, 0) [draw, rounded corners, fill=gray!10, minimum width=2.8cm, minimum height=1cm, align=center] {Server端};

% 网络通信子系统
\node (net) at (0, -1.8) [draw, rounded corners, fill=blue!5, minimum width=9.5cm, minimum height=1cm, align=center]
{网络通信子系统(Socket,C/S 架构)};

% 会话管理模块
\node (session) at (0, -3.8) [draw, rounded corners, fill=yellow!10, minimum width=9.5cm, minimum height=1.2cm, align=center]
{会话管理 / 并发 / 互斥模块\\(多线程、用户锁定、状态管理)};

% 核心业务子系统:用户、订单、商品
\node (user) at (-5, -6.2) [draw, rounded corners, fill=green!10, minimum width=3.2cm, minimum height=1cm, align=center] {用户子系统};
\node (order) at (0, -6.2) [draw, rounded corners, fill=green!10, minimum width=4.2cm, minimum height=1cm, align=center] {订单与购物车子系统};
\node (product) at (5, -6.2) [draw, rounded corners, fill=green!10, minimum width=3.2cm, minimum height=1cm, align=center] {商品子系统};

% 数据存储子系统(底部中心)
\node (data) at (0, -9.0) [draw, rounded corners, fill=red!5, minimum width=11cm, minimum height=1.2cm, align=center]
{数据存储子系统\\(用户、商品、订单、购物车、会话锁等持久化)};

% 连线
\draw[->, thick] (client) -- (net);
\draw[->, thick] (server) -- (net);
\draw[->, thick] (net) -- (session);
\draw[->, thick] (session) -- (order);

\draw[<->, thick] (user) -- (order);
\draw[<->, thick] (order) -- (product);

\draw[->, thick] (user.south) -- ++(0,-1) -- (data.north -| user.south);
\draw[->, thick] (order.south) -- ++(0,-1) -- (data.north -| order.south);
\draw[->, thick] (product.south) -- ++(0,-1) -- (data.north -| product.south);

\end{tikzpicture}
\caption{系统架构分层与子系统关系图}
\label{fig:system-architecture}
\end{figure}



\newpage
各子系统之间层次分明,模块边界清晰,既便于扩展升级,又便于维护和测试。
\newpage

\section{详细设计}

\subsection{用户子系统详细设计}

用户子系统负责平台的用户注册、登录、权限区分、信息维护等核心功能。其主要任务是实现对用户身份、账户状态、用户信息(如用户名、密码、余额、用户类型等)的管理,并支持用户的认证、密码修改、余额充值等操作。

\vspace{0.5em}
\textbf{主要类及关系:}
\begin{itemize}
    \item \textbf{User(用户基类)}:定义了所有用户共有的属性和操作,如用户名、密码、余额、登录接口、权限判断等,是后续派生类的公共基类。
    \item \textbf{Customer(顾客类)}:继承自User,代表普通顾客,拥有购物车管理、下单、订单支付等功能。
    \item \textbf{Seller(商家类)}:继承自User,代表平台商家,拥有商品上架、库存管理、折扣设置、销售管理等功能。
    \item \textbf{UserManager(用户管理器)}:负责所有用户的注册、登录、查找、用户数据的持久化与加载(文件读写),以及用户集合的统一管理。对外提供用户身份校验、注册、密码修改、余额修改、用户数据保存等接口,是连接系统其余子模块(如订单、商品管理等)的桥梁。
\end{itemize}

\vspace{0.5em}
\textbf{类图与关系示意:}
\begin{figure}[htbp]
\centering
\begin{tikzpicture}[node distance=1.8cm, >=latex, font=\small]
    % 类
    \node (um) [draw, rounded corners, fill=green!10, minimum width=2.8cm, align=center] {UserManager};
    \node (user) [below left=1.4cm and 2.2cm of um, draw, rounded corners, minimum width=2.8cm, align=center] {User(抽象基类)};
    \node (cust) [below=1.3cm of user, draw, rounded corners, minimum width=2.8cm, align=center] {Customer};
    \node (sell) [below=1.3cm of user, draw, rounded corners, right=2.2cm of cust, minimum width=2.8cm, align=center] {Seller};

    % 关系
    \draw[->, thick] (um) -- node[left,align=center] {\scriptsize 管理/注册/登录\\\scriptsize 持久化} (user);
    \draw[->, thick] (cust) -- (user);
    \draw[->, thick] (sell) -- (user);

    % 说明
    \node [above=0.3cm of um, align=center, font=\small] {用户子系统主要类结构};
\end{tikzpicture}
\caption{用户子系统关系图}
\label{fig:user-system-architecture}
\end{figure}

\vspace{0.5em}
\textbf{主要接口说明:}
\begin{itemize}
    \item \texttt{UserManager::RegisterUser()}:注册新用户,校验用户名唯一性并支持类型选择(顾客/商家)。
    \item \texttt{UserManager::Login()}:用户登录验证,检查用户名和密码是否匹配。
    \item \texttt{UserManager::SaveToFile() / LoadFromFile()}:将所有用户数据持久化到文件或从文件恢复。
    \item \texttt{User::CheckPassword()}:密码校验。
    \item \texttt{User::Recharge()}:余额充值。
    \item \texttt{User::SetPassword()}:修改密码。
\end{itemize}

\vspace{0.5em}
用户子系统通过 \texttt{UserManager} 实现了所有用户的统一管理、认证、状态维护和信息持久化,是平台安全、账户分离、权限控制和用户体验的基础。其对外接口为订单、购物车、商品子系统等提供了身份支撑与数据保障。

\subsection{商品子系统详细设计}

商品子系统负责平台所有商品的统一管理,包括商品的上架、展示、搜索、库存调整、折扣设置等功能。该子系统为用户(顾客/商家)提供了商品浏览、搜索、购买等基础服务,同时支持商家对其商品的维护操作。

\vspace{0.5em}
\textbf{主要类及关系:}
\begin{itemize}
    \item \textbf{Product(商品基类)}:抽象类,包含商品的基本属性(如名称、描述、类别、单价、库存、拥有者、商品编号等)和通用操作。派生出具体商品类型。
    \item \textbf{Book / Food / Clothes(具体商品类)}:分别代表图书、食品、服装,继承自Product。可扩展更多类别,每类可有自身特性或重载方法。
    \item \textbf{ProductManager(商品管理器)}:负责所有商品对象的管理、搜索、增删改查、库存与折扣管理、商品数据的持久化与加载(文件读写),以及商品与卖家、订单之间的关联。
\end{itemize}

\vspace{0.5em}
\textbf{类图与关系示意:}
\begin{figure}[htbp]
\centering
\begin{tikzpicture}[node distance=1.5cm, >=latex, font=\small]
    % 类
    \node (pm) [draw, rounded corners, fill=green!10, minimum width=2.9cm, align=center] {ProductManager};
    \node (product) [below=1.3cm of pm, draw, rounded corners, minimum width=2.9cm, align=center] {Product(抽象基类)};
    \node (book) [below left=0.8cm and 2.1cm of product, draw, rounded corners, minimum width=2.2cm, align=center] {Book};
    \node (food) [below=1.3cm of product, draw, rounded corners, minimum width=2.2cm, align=center] {Food};
    \node (clothes) [below right=0.8cm and 2.1cm of product, draw, rounded corners, minimum width=2.2cm, align=center] {Clothes};

    % 关系
    \draw[->, thick] (pm) -- node[right,align=left] {\scriptsize 管理/查找/\\\scriptsize 库存/折扣/持久化} (product);
    \draw[->, thick] (book) -- (product);
    \draw[->, thick] (food) -- (product);
    \draw[->, thick] (clothes) -- (product);

    % 说明
    \node [above=0.3cm of pm, align=center, font=\small] {商品子系统主要类结构};
\end{tikzpicture}
\caption{商品子系统关系图}
\label{fig:product-system-architecture}
\end{figure}

\vspace{0.5em}
\textbf{主要接口说明:}
\begin{itemize}
    \item \texttt{ProductManager::AddProduct()}:上架新商品,指定类别、名称、描述、价格、库存等信息。
    \item \texttt{ProductManager::SearchByName()}:支持按名称或关键词搜索商品,返回匹配集合。
    \item \texttt{ProductManager::GetProductByName()}:根据名称查找单个商品对象。
    \item \texttt{ProductManager::ApplyDiscount()}:为指定类别商品批量设置折扣率。
    \item \texttt{ProductManager::SaveToFile() / LoadFromFile()}:商品数据的持久化与恢复。
    \item \texttt{Product::SetStock()/GetStock()}:设置/获取商品库存。
    \item \texttt{Product::SetOwner()/GetOwner()}:设置/获取商品所有者(商家)。
\end{itemize}

\vspace{0.5em}
商品子系统通过 \texttt{ProductManager} 实现所有商品的集中存储、查找、展示和库存、折扣等管理,确保了商品信息的完整性、一致性和高效可用。系统各业务模块(如用户子系统、订单与购物车子系统)均可通过该子系统接口访问和操作商品数据。

\subsection{订单与购物车子系统详细设计}

订单与购物车子系统负责平台中顾客的购物行为流程,包括商品加入购物车、购物车管理、订单生成、订单支付、订单状态变更等功能。该子系统直接关联商品子系统和用户子系统,是交易流程的核心实现部分。

\vspace{0.5em}
\textbf{主要类及关系:}
\begin{itemize}
    \item \textbf{CartItem(购物车条目类)}:表示购物车中单一商品条目,包含商品指针、数量、价格等信息。
    \item \textbf{ShoppingCart(购物车类)}:维护顾客当前购物车所有商品及其数量,提供添加、删除、更新、清空等操作,并支持购物车内容的持久化。
    \item \textbf{Order(订单类)}:封装一次购物流程的全部信息,包括订单编号、顾客、商品明细(CartItem集合)、订单总金额、状态(未支付、已支付、已取消)等。
    \item \textbf{OrderManager(订单管理器)}:负责所有订单的创建、支付、取消、加载、保存等管理操作。订单的生命周期、支付逻辑、商家分账、库存变更等也由此类协调完成。
\end{itemize}

\vspace{0.5em}
\textbf{类图与关系示意:}
\begin{figure}[htbp]
\centering
\begin{tikzpicture}[node distance=1.4cm, >=latex, font=\small]
    % 类
    \node (om) [draw, rounded corners, fill=green!10, minimum width=3.1cm, align=center] {OrderManager};
    \node (order) [below=1.3cm of om, draw, rounded corners, minimum width=2.8cm, align=center] {Order};
    \node (cart) [left=3.3cm of order, draw, rounded corners, minimum width=2.5cm, align=center] {ShoppingCart};
    \node (item) [below=1.1cm of cart, draw, rounded corners, minimum width=2.2cm, align=center] {CartItem};

    % 关系
    \draw[->, thick] (om) -- node[right,align=left] {\scriptsize 订单创建/管理/支付} (order);
    \draw[->, thick] (order) -- node[above,align=center] {\scriptsize 商品明细} (item);
    \draw[->, thick] (cart) -- node[right,align=left] {\scriptsize 购物车条目} (item);

    % 说明
    \node [above=0.3cm of om, align=center, font=\small] {订单与购物车子系统主要类结构};
\end{tikzpicture}
\caption{订单与购物车子系统关系图}
\label{fig:order-and-cart--system-architecture}
\end{figure}

\subsection{网络子系统详细设计}

网络子系统负责实现平台客户端与服务器端之间的通信,采用传统的 C/S 架构,通过 Socket 编程进行消息收发,支持多用户并发访问、会话管理、错误处理等功能。该子系统是平台从单机模式升级到网络分布式的重要基础,保障了业务模块的远程调用和数据同步。

\vspace{0.5em}
\textbf{主要功能与职责:}
\begin{itemize}
    \item 负责客户端与服务器之间的连接建立、断开与管理。
    \item 使用 TCP Socket 实现可靠的数据传输,采用 JSON 作为消息格式,确保跨平台数据结构解析。
    \item 支持多线程并发处理多个客户端会话,保证各用户操作互不干扰。
    \item 实现用户登录互斥(防止同一用户多端同时登录)、会话状态同步、用户锁定等机制。
    \item 提供底层 SocketStream 抽象,屏蔽底层网络细节,对业务模块透明,提升代码复用与健壮性。
    \item 支持通信异常、网络中断、非法请求等多种错误场景处理,保障系统鲁棒性。
\end{itemize}

\vspace{0.5em}
\textbf{主要类及结构:}
\begin{itemize}
    \item \textbf{SocketStream}:自定义的网络通信封装类,负责 socket 的创建、连接、监听、读写数据(支持按行、按块收发),并提供初始化、关闭等操作接口,内部自动管理资源释放。
    \item \textbf{客户端/服务器主循环}:服务器端主进程监听端口、接受新连接,为每个会话分配独立线程,由线程执行 Handle 逻辑进行完整一次业务交互,线程退出时释放资源。客户端主循环负责用户输入、命令解析、与服务器 JSON 消息收发、结果展示等。
    \item \textbf{会话互斥模块}:采用会话锁文件方式(如 \texttt{data/sessions/username.lock}),防止同一账号多端登录,退出/异常时自动释放锁,保证会话一致性和安全性。
\end{itemize}

\vspace{0.5em}
\textbf{通信协议说明:}
\begin{itemize}
    \item \textbf{底层协议}:采用 TCP/IP 进行数据传输,保证消息可靠送达。
    \item \textbf{消息格式}:统一采用 JSON 格式进行请求与响应的编解码,字段包括 type(请求类型)、data(具体数据)、ok(结果标识)、msg(错误提示)等。
    \item \textbf{业务协议}:平台所有操作(如登录、注册、查询、购物、订单管理等)均以 JSON 消息包形式在客户端和服务器之间双向流转,详见接口协议说明章节。
\end{itemize}

\vspace{0.5em}
网络子系统为平台提供了高效、可靠、可扩展的通信能力,实现了分布式部署与远程访问,为用户和业务逻辑子系统之间的交互建立了桥梁。其高鲁棒性的并发处理与错误管理机制,显著提升了平台的实际可用性与工程质量。


\subsection{数据库说明}

本系统采用结构化文本文件作为数据持久化介质,模拟关系型数据库表的存储与操作。每类业务数据采用独立的文本文件存储,每行表示一条数据记录,字段间以空格或竖线等分隔符分隔。各管理模块在内存中实现相应“表”的增删改查功能。下表详细列出各文件的数据结构与字段含义:

\vspace{0.5em}
\textbf{1. 用户数据表(users.txt)}
\begin{center}
\begin{tabular}{|c|c|c|c|c|}
\hline
\textbf{字段名} & \textbf{类型} & \textbf{说明} & \textbf{举例} \\
\hline
username & string & 用户名,唯一 & Coco \\
password & string & 密码(明文) & coco \\
balance  & double & 账户余额    & 3400.0 \\
userType & string & 用户类型(Seller/Customer) & Customer \\
\hline
\end{tabular}
\end{center}

\vspace{0.5em}
\textbf{2. 商品数据表(products.txt)}
\begin{center}
\begin{tabular}{|c|c|c|c|c|}
\hline
\textbf{字段名} & \textbf{类型} & \textbf{说明} & \textbf{举例} \\
\hline
category    & string & 商品类别      & Food \\
name        & string & 商品名称      & Milk \\
desc        & string & 商品描述      & Pure \\
price       & double & 单价         & 10.0 \\
stock       & int    & 库存数量      & 100 \\
owner       & string & 所有者用户名   & Admin \\
productId   & string & 商品唯一编号   & Food-0 \\
\hline
\end{tabular}
\end{center}

\vspace{0.5em}
\textbf{3. 订单数据表(orders.txt)}
\begin{center}
\begin{tabular}{|c|c|c|c|c|}
\hline
\textbf{字段名} & \textbf{类型} & \textbf{说明} & \textbf{举例} \\
\hline
orderId    & string & 订单编号          & ORD-000000 \\
customer   & string & 顾客用户名        & Coco \\
itemCount  & int    & 商品项数          & 1 \\
orderTime  & int    & 下单时间(时间戳) & 1749480442 \\
total      & double & 订单总金额        & 200.0 \\
status     & string & 订单状态(Paid/Unpaid/Failed) & Paid \\
\hline
\end{tabular}
\end{center}
订单明细项:
\begin{center}
\begin{tabular}{|c|c|c|c|}
\hline
\textbf{字段名} & \textbf{类型} & \textbf{说明} & \textbf{举例} \\
\hline
productId  & string & 商品编号   & Food-0 \\
quantity   & int    & 数量       & 20 \\
\hline
\end{tabular}
\end{center}

\vspace{0.5em}
\textbf{4. 购物车数据表(carts/[username].txt)}
\begin{center}
\begin{tabular}{|c|c|c|}
\hline
\textbf{字段名} & \textbf{类型} & \textbf{说明} \\
\hline
productId  & string & 商品编号 \\
quantity   & int    & 数量 \\
\hline
\end{tabular}
\end{center}

\vspace{0.5em}
\textbf{5. 会话锁定(sessions/[username].lock)}

\begin{center}
\begin{tabular}{|c|c|c|}
\hline
\textbf{字段名} & \textbf{类型} & \textbf{说明} \\
\hline
lockFlag & string & 锁标志,内容为"LOCKED"时表示账号已登录 \\
\hline
\end{tabular}
\end{center}

\vspace{0.5em}
\textbf{备注:}
\begin{itemize}
    \item 所有数据文件均为结构化文本文件,便于人工审阅与调试。
    \item 如需升级为关系型数据库,可按上述字段设计建表语句,数据类型可直接映射为 SQL 标准类型(VARCHAR/DOUBLE/INT 等)。
    \item 每次数据变更操作(注册、上架、下单、支付等)均立即同步持久化,确保数据一致性和可靠性。
\end{itemize}

\vspace{0.5em}
通过该数据存储子系统,平台能够高效、安全、结构化地管理所有业务核心数据,支撑系统稳定运行和用户良好体验。

\subsection{接口协议说明}

本系统采用自定义的应用层协议,客户端与服务器端之间的通信全部基于 TCP Socket,通过 JSON 格式的消息进行数据封装与解析。所有请求与响应均为结构化的 JSON 对象,保证了协议的跨平台兼容性和良好的可扩展性。

\vspace{0.5em}
\textbf{底层承载协议:}
\begin{itemize}
    \item 采用 TCP/IP 协议作为通信的底层承载,保障消息可靠送达和顺序一致性。
    \item 客户端与服务器通过 Socket 建立连接,所有业务数据通过该连接收发。
\end{itemize}

\vspace{0.5em}
\textbf{消息格式规范:}
\begin{itemize}
    \item 所有请求与响应均采用 JSON 文本格式,每个消息为一行字符串,通过 SocketStream 的 SendLine/RecvLine 实现。
    \item 通用消息结构如下:
\begin{verbatim}
{
  "type": "请求类型",
  "data": { ... }
}
\end{verbatim}
    \item 响应消息增加如下字段:
\begin{verbatim}
{
  "ok": true/false,     // 请求是否成功
  "data": { ... },      // 返回结果数据
  "msg":  "错误提示"    // 可选,出错时携带错误原因
}
\end{verbatim}
\end{itemize}

\vspace{0.5em}
\textbf{典型协议数据单元说明:}

\begin{itemize}
    \item \textbf{注册请求/响应}
\begin{verbatim}
请求: { "type": "Register", "data": { "username": "...", "password": "...", "userType": "Customer/Seller" } }
响应: { "ok": true/false, "msg": "..." }
\end{verbatim}

    \item \textbf{登录请求/响应}
\begin{verbatim}
请求: { "type": "Login", "data": { "username": "...", "password": "..." } }
响应: { "ok": true/false, "data": { "userType": "Customer/Seller" }, "msg": "..." }
\end{verbatim}

    \item \textbf{商品查询/响应}
\begin{verbatim}
请求: { "type": "List", "data": {} }
响应: { "ok": true, "data": [ { "cat": "...", "name": "...", "desc": "...", "price": ..., "stock": ..., "owner": "..." }, ... ] }
\end{verbatim}

    \item \textbf{购物车管理、订单、支付等}  
所有请求均类似,"type" 字段标明操作类型,"data" 携带操作参数。所有响应都带 "ok" 字段指示结果。
\end{itemize}

\vspace{0.5em}
\textbf{协议语义说明:}
\begin{itemize}
    \item 客户端每次发送操作请求,服务端解析 type 字段分派到相应处理逻辑,处理结果打包成 JSON 响应返回。
    \item 出错时,"ok" 置为 false,"msg" 字段详细说明原因。
    \item 所有操作请求与响应均为一问一答模式,保证了状态一致性和时序逻辑清晰。
    \item 连接建立后保持长连接,直到用户显式退出或网络断开。
\end{itemize}

\vspace{0.5em}
\textbf{补充:}
\begin{itemize}
    \item 所有通信协议细节在客户端/服务器的网络主循环及 SocketStream 模块内实现,业务代码与网络代码严格解耦。
    \item 如需协议升级,仅需约定 JSON 字段,不影响底层 Socket 通信及主业务逻辑。
\end{itemize}

\newpage
\section{实现}

\subsection{实验过程中遇到的主要问题和解决方案}

在系统开发过程中,遇到如下主要问题,并逐一进行了有效的分析和解决:

\begin{enumerate}
    \item \textbf{并发会话与数据一致性} \\
    服务器端需要支持多用户并发访问,且保证数据一致性。采用多线程为每个客户端分配独立会话,所有用户管理、订单管理、商品管理等核心数据模块均加锁处理(如 \texttt{std::mutex} 保护用户/订单容器),防止并发修改导致的数据错乱。

    \item \textbf{用户多端登录互斥机制} \\
    平台要求同一账号不能同时在多个终端登录。为此,设计了基于 lock 文件的互斥机制(如 \texttt{data/sessions/[username].lock})。登录前检测锁文件是否存在,若存在则拒绝登录,登出或异常断开时自动删除锁文件,防止“僵尸会话”。

    \item \textbf{数据持久化与文件格式解析} \\
    初始阶段由于没有关系型数据库,全部数据通过文本文件存储。为保证可靠性和可扩展性,所有存取均采用结构化格式,分隔符严格规定。针对文件读写的各种边界条件(如数据缺失、重复、非法字符),逐步补充了健壮的输入校验和异常处理,确保系统在异常输入下也能稳定运行。

    \item \textbf{客户端/服务器端界面友好性与输入健壮性} \\
    为保证用户体验,客户端所有输入均有详细提示,非法输入自动重试,关键功能均有明确错误提示(如余额不足、商品不存在、库存不足、重复下单等),保证系统使用流畅、易懂。

    \item \textbf{订单与购物车数据同步和自动清理机制} \\
    用户支付后购物车内容应自动清空。通过在支付/下单逻辑中加入清理调用,同时对购物车持久化文件进行同步删除或覆盖,避免历史遗留数据干扰用户后续操作。

    \item \textbf{断线重连与数据恢复能力} \\
    系统支持断线后重新登录,购物车数据自动从持久化文件恢复,未支付订单保留,保障了良好的用户体验和平台健壮性。
\end{enumerate}

综上,针对实验过程中遇到的各类典型问题,均通过模块化设计、异常处理、数据校验、互斥机制等手段实现了高质量、健壮的系统架构,保证了多用户环境下的正确性和易用性。

\subsection{想法}

在本系统的设计与实现过程中,深刻体会到模块化、面向对象编程和工程化思想的重要性。通过将用户、商品、订单、购物车等业务逻辑进行清晰划分和解耦,大大提升了系统的可维护性与可扩展性。采用 C/S 架构和 Socket 通信,既锻炼了网络编程能力,也使系统具备了多用户并发和分布式部署的基础。在数据存储方面,通过类比数据库表设计结构化文本文件,为后续系统升级和迁移数据库打下良好基础。

\subsection{经验}

\begin{itemize}
    \item 模块划分清晰、接口设计合理,可以有效降低后续功能扩展和维护成本。
    \item 持久化机制要与内存模型保持同步,关键节点须及时保存和恢复,避免数据丢失。
    \item 多线程与多用户并发访问时,必须严格加锁,提前考虑数据一致性问题。
    \item 错误处理和输入校验不容忽视,友好的用户交互与明确的错误提示能极大提升体验。
    \item 开发过程中,先实现单机功能再逐步升级为网络版,是比较稳妥的策略。
\end{itemize}

\subsection{教训}

\begin{itemize}
    \item 初期未充分考虑并发与异常场景,导致后续补充加锁和错误处理时较为繁琐,说明系统设计时需有前瞻性。
    \item 用户购物车的持久化和订单生成过程涉及多步操作,为避免“脏数据”或部分提交,所有涉及资金扣除、库存扣减、订单生成等步骤均严格顺序执行,操作失败时及时回滚或报错,保证业务原子性。
\end{itemize}

\subsection{致谢}

\vspace{0.5em}
在此由衷感谢双锴老师和助教老师的悉心指导与认真负责的验收。他们不断敦促我们关注工程细节、重视规范,同时也鼓励和启发我们思考更优的系统架构与实现方案。老师们的高标准、严要求使我收获颇丰,也为日后更复杂、安全性要求更高的开发打下了坚实基础。

% \newpage
% \appendix
% \section*{附录:源代码清单}
% \addcontentsline{toc}{section}{附录:源代码清单}

% \subsection*{附录 A:Selective Repeat 协议实现(SelectiveRepeat.c)}
% \lstinputlisting[language=C, caption={./SelectiveRepeat.c -- Selective Repeat 协议源代码}, breaklines=true]{SelectiveRepeat.c}

% \subsection*{附录 B:Go-Back-N 协议实现(GoBackN.c)}
% \lstinputlisting[language=C, caption={./GoBackN.c -- Go-Back-N 协议源代码}, breaklines=true]{GoBackN.c}

\end{document}

